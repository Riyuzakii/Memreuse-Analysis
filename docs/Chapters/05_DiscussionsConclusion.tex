% Discussions & Conclusion

The obtained plots of cumulative distribution function of access distances for \textit{prog\{1, 2, 3, 4\}.c} show subtle yet noticeable differences when taken the original trace and after modelling through 2MB 16-way single level cache and focussing only on miss traces.

\begin{enumerate}
	\item The cdf-plots for original trace have uneven slope while the cdf-plots for miss traces shows a 
	proper \textit{sigmoidal} curve, which is generally seen for \textit{(Gaussian)}\textit{ Normal distribution}. 
	This implies even if the program has uneven access pattern, the hit/miss pattern solely depends on cache 
	specifications and is more-or-less similar for any program, on a given trace.
	
	\item The initial slope of cdf-plot for miss trace is \textit{zero} and increase gradually. While for the
	original plots, there is a sudden increase initially which evens out in the end. This can be explained
	due to \textit{temporal locality} of accesses, where programs access nearby elements quickly in a short
	duration which results in high number of low-access-distance patterns.
	
	\item The slope for cdf-plots of miss traces starts out at \textit{zero} because it is expected that for
	a reasonable cache specification, low-access-distance for a block will generally result in a hit. That is,
	misses will be observed for blocks with high-access-distance as the block might get evicted, following
	the eviction policies like LRU.
	
	\item The cdf-plot for miss traces show a sudden jump midway (observed in Gaussian curves). This can be
	attributed to the fact that there is a certain limit (e.g $6 \times 10^{4}$) beyond which the cache 
	capacity is exceeded and we see a large number of misses. This lets us know about the critical access
	distance of the blocks and that accesses below it have a greater chance of registering a hit.
\end{enumerate}