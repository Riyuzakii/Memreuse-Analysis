% Discussions & Conclusion
    Following are some of the observations on the simulation results.
\subsection{Part 1}
\subsubsection{L2 Hits/Misses (Exclusive) == L2 Hits/Misses (NINE)}
This is observed because at any point in time, the addresses that are present in case of Exclusive 
policy are same as that of NINE, albeit in different configuration. We can analyze all 
possibilities to prove this:
\begin{itemize}
	\item \textbf{L2 Hit:} Nothing changes here.
	\item \textbf{L3 Hit:} The block gets added to L2 Cache. Although the block is invalidated from L3 in exclusive case.
	\item \textbf{L3 Miss:} The block is brought from memory and added to both L2 and L3 cache.
\end{itemize}
Hence, in all the cases, if the configuration is same after \textit{kth} steps, it will be same 
after \textit{(k+1)th} step.

\subsubsection{L3 Misses: Inclusive $ >(\approx)$ NINE $>>$ Exclusive}
At any given point, L3 cache in case of Exclusive policy will have most absolute space (due to 
non-similarity of data in L2 and L3). This explains why L3 misses in Exclusive are significantly 
lesser.\\
There isn't much difference in L3 misses for Inclusive and NINE cases, although Inclusive is 
slighly more. This can be attributed to the fact that eviction from L3 also evicts from L2 for 
Inclusive policy, which is not the case with NINE. So, an access to a block evicted from L3 would
encounter a hit in L2 for NINE but would unnecessarily miss in both L2, L3 in Inclusive case.

\subsubsection{L2 Misses: Inclusive $>$ Exclusive == NINE}
This is a serious problem with Inclusive policy, as mentioned in problem statement. If a block 
recieves constant hits in L2 cache, its rank in the LRU indexing will be quite high. As a result,
the block will normally never be evicted from L2. But its age in L3 will drop significantly and
 might even become victim to an eviction. Following the Inclusive policy, it will also have to be
removed from L2 which would unnecessarily increase L2 misses.

\subsection{Part 2}
\subsubsection{Method of categorizing FA L3 misses}
\textbf{Cold:} Number of unique block references.\\
\textbf{Capacity:} FA-L3 misses - Cold misses.\\
\textbf{Conflict:} 16-way L3 misses - FA-L3 misses.

\begin{enumerate}
	\item Number of capacity misses for LRU eviction policy are more than that of Belady's policy since Belady is the most optimal solution for block replacement.
	\item Number of conflict misses for \textit{sphinx3} trace are negative (-179886). For some uncommon cases, the approach of calculating conflict misses from cold and capacity of fully-associative cache can result in a negative number \cite{solihin2015fundamentals}.
\end{enumerate}